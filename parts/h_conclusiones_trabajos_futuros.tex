\section{Conclusiones y trabajos futuros}

El presente trabajo tuvo como objetivo analizar, implementar y comparar distintas técnicas de sobremuestreo en contextos de clasificación con datos desbalanceados. Se evaluaron métodos tradicionales como SMOTE, ADASYN y Borderline-SMOTE, así como variantes recientes como PC-SMOTE, AR-ADASYN y α‑DBASMOTE, tanto en sus versiones individuales como combinadas.

Los resultados obtenidos confirmaron que las técnicas avanzadas, al incorporar criterios de densidad, riesgo y pureza, ofrecieron mejoras sustanciales en métricas como F1-score y AUC-ROC. En particular, la técnica híbrida α‑DBASMOTE + AR‑ADASYN logró un desempeño superior en varios escenarios, combinando una selección precisa de muestras críticas con una generación adaptativa de ejemplos sintéticos.

Además, se observó que el rendimiento final dependió no solo del método de sobremuestreo, sino también del clasificador utilizado y del dataset evaluado. Esta dependencia resalta la importancia de realizar evaluaciones integrales que contemplen múltiples combinaciones de técnicas y clasificadores.

Como trabajos futuros, se identificaron varias líneas de exploración:

\begin{itemize}
  \item Incorporar estrategias de selección de atributos para mejorar la calidad de las instancias generadas.
  \item Extender el análisis a datasets multietiqueta o secuenciales, donde el problema del desbalance presenta nuevas complejidades.
  \item Aplicar los algoritmos propuestos en dominios reales, como imágenes médicas o series temporales, evaluando su desempeño en entornos no controlados.
  \item Diseñar variantes más eficientes computacionalmente que mantengan la precisión sin aumentar el costo de entrenamiento.
\end{itemize}

En síntesis, esta investigación aportó evidencia empírica y formal sobre la eficacia de nuevas estrategias de sobremuestreo, destacando su potencial para mejorar la clasificación en dominios complejos y desbalanceados.
