\section*{Pseudocódigo: \textit{PC\_SMOTE}}
\noindent\hypertarget{pseudocodigo_pc_smote}{}\hyperlink{toc}{\small$\uparrow$ Volver al índice}

\begin{pseudo}[Pseudocódigo PC\_SMOTE]
Entrada: conjunto de datos desbalanceado, vecinos k, umbral de percentil \( p \)  
Salida: nuevas muestras sintéticas

Para cada instancia minoritaria \( x_i \):  
    Calcular k vecinos más cercanos  
    Calcular distancias inversas \( \alpha_j = 1 / \text{dist}(x_i, x_j) \)  
    Sumar pesos para vecinos mayoritarios \( \alpha'_n \)  
Fin Para

Calcular percentil \( p \) de los valores \( \alpha'_n \)  
Filtrar las muestras cuya \( \alpha'_n \) esté por debajo del percentil  

Para cada muestra filtrada:  
    Generar \( g_i \) puntos sintéticos entre \( x_i \) y vecinos minoritarios  
    Para \( j = 1 \) hasta \( g_i \):  
        Seleccionar aleatoriamente vecino minoritario \( x_{nn} \)  
        Calcular: \( x^{(j)}_{\text{syn}} = x_i + \lambda \cdot (x_{nn} - x_i) \), con \( \lambda \in [0, 1] \)  
    Fin Para  
Fin Para
\end{pseudo}