\section*{Pseudocódigo del algoritmo αDistance Borderline-ADASYN-SMOTE}
\noindent\hyperlink{toc}{\small$\uparrow$ Volver al índice}

\textbf{Referencia:} Feng \& Li (2021)

\begin{algorithm}[H]
\caption{αDistance Borderline-ADASYN-SMOTE}
\begin{algorithmic}[1]
\State \textbf{Input:} Conjunto de datos $\mathcal{D}$, número de vecinos $m$, proporción deseada de balance $\beta \in [0, 1]$
\Statex

\For{cada muestra minoritaria $p_i \in \mathcal{D}_{min}$}
    \State Obtener sus $m$ vecinos más cercanos.
    \State Separar los vecinos en:
    \begin{itemize}
        \item Minoritarios: cantidad $pnum$
        \item Mayoritarios: cantidad $nnum$
    \end{itemize}
    \For{cada vecino $p_j$}
        \State Calcular peso: $\alpha_j = \frac{1}{\text{dist}(p_i, p_j)}$
    \EndFor
    \State Calcular sumatoria de pesos:
    \begin{itemize}
        \item $\alpha'_p = \sum \alpha_j$ de vecinos minoritarios
        \item $\alpha'_n = \sum \alpha_j$ de vecinos mayoritarios
    \end{itemize}
    \If{$\alpha'_n > \alpha'_p$}
        \State Marcar $p_i$ como muestra peligrosa
    \EndIf
\EndFor

\vspace{1em}
\State Calcular el total de ejemplos sintéticos a generar: $G = (N - n) \cdot \beta$
\Statex Donde $N$ es la cantidad de muestras mayoritarias y $n$ la cantidad de minoritarias

\vspace{1em}
\For{cada muestra peligrosa $p_i$}
    \State Calcular: $r_i = \frac{\Delta_i}{m}$
    \State Normalizar: $\hat{r}_i = \frac{r_i}{\sum r_i}$
    \State Asignar: $g_i = \hat{r}_i \cdot G$
\EndFor

\vspace{1em}
\For{cada muestra peligrosa $p_i$}
    \For{$g_i$ veces}
        \State Seleccionar un vecino minoritario $p_z$ aleatorio
        \State Generar muestra sintética:
        \[
        s = p_i + \lambda \cdot (p_z - p_i), \quad \lambda \sim \mathcal{U}[0, 1]
        \]
    \EndFor
\EndFor
\end{algorithmic}
\end{algorithm}

